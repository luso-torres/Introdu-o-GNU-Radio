\section{Exemplos com código} % Seções são adicionadas para organizar sua apresentação em blocos discretos, todas as seções e subseções são automaticamente exibidas no índice como uma visão geral da apresentação, mas NÃO são exibidas como slides separados.


%----------------------------------------------------------------------------------------
    \begin{frame}[fragile]
    \frametitle{Computing the Histrogram}
    \begin{columns}

         \column{0.48\textwidth}  % First column
    \begin{roundedcodebox}
    \begin{python}
#-----------------------------------
##   Compute a histogram of an image.
#   FAON - Jan 14th, 2023.
#-----------------------------------
def Compute_an_image_histogram(image, Bins):          
    N, M = image.shape  
    histogram = np.zeros((256), dtype = int)             
    histogram_Bins = np.zeros((Bins), dtype = int)
    if( Bins < 256):
        bins_step = 256.0 / Bins        
        for n in range(N):
          for m in range(M):
            histogram[image[n, m]]+= 1   
    \end{python}
\end{roundedcodebox}
   \column{0.48\textwidth}  % First column
    \begin{roundedcodebox}
    \begin{python}  
      if( Bins == 256 ): # In that case we have already computed histogram_Bins.
        histogram_Bins = histogram
    else:
        k = 0
        i=0
        while i < 256: 
            while ( (i >= (int) ( k * bins_step + 0.5) ) and (i < (int) (  (k + 1) * bins_step + 0.5 ) )):
                histogram_Bins[k] += histogram[i] 
                i += 1                 
            k +=1                 
    return histogram_Bins    
      \end{python}
\end{roundedcodebox}
    \end{columns}
\end{frame}

%----------------------------------------------------------------------------------------
    \begin{frame}[fragile]
    \frametitle{CDF/Inversa/Probability/Shannon}
    \begin{columns}

         \column{0.48\textwidth}  % First column
    \begin{roundedcodebox}
    \begin{python}
def Compute_cdf(pdf, Bins):
    cdf = np.zeros((Bins), dtype = float)
    cdf[0] = pdf[0]
    for k in range(1, Bins):   
        cdf[k] = cdf[k-1]+pdf[k]
    return cdf  
#----------------------------------
def Compute_inverse_cdf(cdf, probability):
    N = cdf.shape[0]          
    for k in range(N): 
        if(cdf[k] > probability):
            return k
            '''
            if(k > 0):
                return k
            else:
                return 0  
            ''' 
     \end{python}
\end{roundedcodebox}
  \column{0.48\textwidth}  % First column
    \begin{roundedcodebox}
    \begin{python}  
def Compute_probability(pmf, x1, x2, Bins): 
    probability = 0
    if( x2 == Bins):
        x2 = x2 - 1
    for k in range(x1, x2 + 1):
        probability += pmf[k]
    return probability   
#----------------------------------
def Compute_Shannon_entropy(pmf, Bins):
    epsilon = 2.2e-16
    bins_entropy = 0
    for i in range(Bins):
        bins_entropy -= pmf[i] * np.log2(pmf[i] + epsilon)
    return bins_entropy
\end{python}
\end{roundedcodebox}
    \end{columns}
\end{frame}

 


%----------------------------------------------------------------------------------------

    \begin{frame}[fragile]
    \frametitle{Robust Mean Absolute Deviation}
    \begin{columns}

         \column{0.48\textwidth}  % First column
    \begin{roundedcodebox}
    \begin{python}
def Compute_Robust_Mean_Absolute
_Deviation(image):
    N, M = image.shape
    image_p100 = np.percentile(image, 10.0)
    image_p900 = np.percentile(image, 90.0)
    #-----------------------------
    Robust_Mean_Absolute_Deviation = 0  
    percentile_range_10_90_pixels
    _values = np.zeros((N * M), dtype = float)
    k = 0
    for n in range(N):
        for m in range(M):
            if( (image[n,m] >= image_p100) and (image[n, m] <= image_p900) ):                    
     \end{python}
\end{roundedcodebox}
  \column{0.48\textwidth}  % First column
    \begin{roundedcodebox}
    \begin{python}  
                percentile_range_10_90_pixels
                _values[k] = image[n,m]
                k += 1
    if(k == 0):
        k = 1       
    percentile_range_10_90_pixels
    _mean = np.mean(percentile_range_10_90_pixels
    _values)   
    Robust_Mean_Absolute_Deviation = np.sum( abs(percentile_range_10_90_pixels
    _values - percentile_range_10_90_pixels
    _mean), axis = None ) / (float) (k)         
    return Robust_Mean_Absolute_Deviation


\end{python}
\end{roundedcodebox}
    \end{columns}
\end{frame}

%----------------------------------------------------------------------------------------

    \begin{frame}[fragile]
    \frametitle{Split/Statistical}
    \begin{columns}

         \column{0.48\textwidth}  % First column
    \begin{roundedcodebox}
    \begin{python}
def split_population(image, x1, x2):
    N, M = image.shape 
    x_pop = np.zeros((3, N * M), dtype = int)
    k1 = 0
    k2 = 0
    k3 = 0
    for n in range(N):
        for m in range(M):
            if(image[n, m] <= x1):
                x_pop[0, k1] = image[n,m]
                k1 += 1
            elif ( (image[n,m] > x1) and (image[n, m] <= x2) ): 
                x_pop[1, k2] = image[n,m]  
                k2 += 1
            else:
                x_pop[2, k3] = image[n,m]  
                k3 += 1            
    population_size = [k1, k2, k3]  
    return population_size, x_pop 
     \end{python}
\end{roundedcodebox}
  \column{0.48\textwidth}  % First column
    \begin{roundedcodebox}
    \begin{python}  
def calculates_statistical_quantities
_of_a_population(x, x1, x2, N): 
    if( x1 == 0):
        M = x2 - x1 + 1
    else:
       M = x2 - x1          
    histogram = np.zeros((M), dtype = int)  
    #------------------------
    #   Compute the histogram using 256 bins.
    #------------------------
    if( x1 == 0):
        for n in range(N):
            histogram[x[n]] += 1  
    else:
        for n in range(N):
            histogram[x[n] - x1 - 1] += 1 
pmf = histogram/N
\end{python}
\end{roundedcodebox}
    \end{columns}
\end{frame}

%----------------------------------------------------------------------------------------

\begin{frame}[fragile]
    \frametitle{Features}
\begin{columns}
  \column{0.48\textwidth}  % First column
    \begin{roundedcodebox}
    \begin{python}  
    #---------------------------
    # Statistical mean.
    #---------------------------
    sum = 0.0
    for k in range(M):    
         sum +=  (k + x1) * pmf[k]
    mean = sum
    #---------------------------
    # Statistical variance.
    #---------------------------
    sum = 0.0
    for k in range(M):    
         sum +=  ((k + x1 - mean)**2.0) * pmf[k]
    variance = sum    
\end{python}
\end{roundedcodebox}
  \column{0.48\textwidth}  % First column
    \begin{roundedcodebox}
    \begin{python}  
    #---------------------------
    # Statistical skewness.
    #---------------------------
    sum = 0.0
    for k in range(M):    
         sum +=  ((k + x1 - mean)**3.0) * pmf[k]
    central_moment_3 = sum 
    skewness = central_moment_3 /variance**1.5       
    #---------------------------
    # Statistical Kurtosis.
    #---------------------------
    sum = 0.0
    for k in range(M):    
         sum +=  ((k + x1 - mean)**4.0) * pmf[k]
    central_moment_4 = sum 
    kurtosis = central_moment_4 /variance**2.0              
    return mean, variance, skewness, kurtosis   

\end{python}
\end{roundedcodebox}
    \end{columns}
\end{frame}

%----------------------------------------------------------------------------------------

\begin{frame}[fragile]
    \frametitle{Statistical Features - Code Output}
\begin{columns}
  \column{0.48\textwidth}  % First column
    \begin{roundedcodebox}
    \begin{python}  
    def Compute_Statistical_Features(image, Bins):        
    features = []     
    N, M = image.shape  
    image_number_of_pixels = float(N * M) 
    #----------------------------
    # Compute image's hitogram using the developed method.
    #----------------------------
    histogram = Compute_an_image_histogram(image, Bins)        
    #histogram_in_ascending_order = np.sort(histogram)               
    #----------------------------
    # Build the Power Mass Function.
    #----------------------------
    pmf = histogram / image_number_of_pixels
    cdf = Compute_cdf( pmf, Bins)        
\end{python}
\end{roundedcodebox}
  \column{0.48\textwidth}  % First column
    \begin{roundedcodebox}
    \begin{python}  
    #---------------------------
    # It computes statistical features of Type I.
    #---------------------------
    image_min = np.min(image) 
    image_max = np.max(image)
    image_mean = round(np.mean(image), 3) 
    image_median = (int) (np.median(image))
    image_std = round(np.std(image), 3)  
    image_var = round(np.var(image), 3) 
    #---------------------------   
    image_mode = np.argmax(pmf)
    image_skew = round(scipy.stats.skew(image, axis = None), 3)
    image_kurtpsis = round(scipy.stats.kurtosis(image, axis = None),3)
    square_range = round((image_max - image_min)**2, 3)
\end{python}
\end{roundedcodebox}
    \end{columns}
\end{frame}

%----------------------------------------------------------------------------------------

\begin{frame}[fragile]
    \frametitle{Features of Type II}
\begin{columns}
  \column{0.48\textwidth}  % First column
    \begin{roundedcodebox}
    \begin{python}  
    #---------------------------   
    # Compute Shannon entropy 2
    #---------------------------   
    Shannon_entropy = round(Compute_Shannon_entropy(pmf, Bins), 3)
    #---------------------------         
    # Normalized energy.
    #---------------------------   
    image_norm_energy = round((np.sum( image**2, axis = None ) / image_number_of_pixels),3)
    #---------------------------   
    # Root-Mean-Square (RMS) value.
    #---------------------------   
    image_rms = round(np.sqrt(image_norm_energy),3)  
\end{python}
\end{roundedcodebox}
  \column{0.48\textwidth}  % First column
    \begin{roundedcodebox}
    \begin{python}  
    #---------------------------   
    # Compute the statistics Features of Type II.
    #---------------------------   
           
    #---------------------------   
    # Compute the percentiles from image using Numpy.
    #---------------------------   
    image_p75 = np.percentile(image, 7.5)
    image_p150 = np.percentile(image, 15.0)      
    image_p850 = np.percentile(image, 85.0)        
    image_p925 = np.percentile(image, 92.5)  
\end{python}
\end{roundedcodebox}
    \end{columns}
\end{frame}

%----------------------------------------------------------------------------------------
\begin{frame}[fragile]
    \frametitle{Features of Type III}
\begin{columns}
  \column{0.48\textwidth}  % First column
    \begin{roundedcodebox}
    \begin{python}  
    # It computes statistical features of Type III.
    #---------------------------  
      
    #---------------------------
    # Compute image interquartile range.
    #---------------------------
    image_p250 = np.percentile(image, 25.0)
    image_p750 = np.percentile(image, 75.0)      
    image_interquartile_range = image_p750 - image_p250
    #---------------------------
    # Mean Absolute Deviation from the mean.
    #---------------------------
    image_abs_deviation_from_the_mean = round((np.sum( np.abs(image - image_mean), axis = None ) / image_number_of_pixels),3)
    #---------------------------
    
\end{python}
\end{roundedcodebox}
  \column{0.48\textwidth}  % First column
    \begin{roundedcodebox}
    \begin{python}  
    # Mean Absolute Deviation from the median.
    #---------------------------
    image_deviation_from_the_median = round((np.sum((image - image_median), axis = None ) / image_number_of_pixels),3)  
    #---------------------------
    # Compute image normalized uniformity from its histogram.
    #---------------------------        
    image_uniformity = round((np.sum( histogram**2, axis = None ) / image_number_of_pixels),3)          
    #---------------------------
    # Compute the Robust Mean Absolute Deviation (rMAD).
    #--------------------------- 
    Robust_Mean_Absolute_Deviation = round(Compute_Robust_Mean_Absolute
    _Deviation(image), 3)
    #---------------------------
   
\end{python}
\end{roundedcodebox}
    \end{columns}
\end{frame}

%----------------------------------------------------------------------------------------
\begin{frame}[fragile]
    \frametitle{Features of Type IV}
\begin{columns}
  \column{0.48\textwidth}  % First column
    \begin{roundedcodebox}
    \begin{python}  
    x1 = Compute_inverse_cdf(cdf, 0.333)
    x2 = Compute_inverse_cdf(cdf, 0.666)
    #---------------------------                
    # Splites the population. 
    #---------------------------      
    population_size, x_pop = split_population(image, x1, x2)
    population_threshold = (int) (image_number_of_pixels/3)
    threshold_deviation_1 = population_size[0] - population_threshold
    threshold_deviation_2 = population_size[1] - population_threshold  
    #threshold_deviation_3 = population_size[2] - population_threshold
    #---------------------------
    # Mean, variance, skeness and kurtosis of population 1.
    #---------------------------
\end{python}
\end{roundedcodebox}
  \column{0.48\textwidth}  % First column
    \begin{roundedcodebox}
    \begin{python}  
    x = x_pop[0,:]
    m1, v1, s1, k1 = calculates_statistical_quantities
    _of_a_population(x, 0, x1, population_size[0])
    m1 = round(m1, 3) 
    v1 = round(v1, 3) 
    s1 = round(s1, 3)     
    k1 = round(k1, 3)   
    #---------------------------
    # It computes mean, variance, skeness and kurtosis of population 2.
    x = x_pop[1,:]
    m2, v2, s2, k2 = calculates_statistical_quantities
    _of_a_population(x, x1, x2, population_size[1]) 
    m2 = round(m2, 3) 
    v2 = round(v2, 3) 
    s2 = round(s2, 3)     
    k2 = round(k2, 3)      
         
\end{python}
\end{roundedcodebox}
    \end{columns}
\end{frame}

%----------------------------------------------------------------------------------------
\begin{frame}[fragile]
    \frametitle{Features of Type IV}
\begin{columns}
  \column{0.48\textwidth}  % First column
    \begin{roundedcodebox}
    \begin{python}  
    #---------------------------
    # It computes mean, variance, skeness and kurtosis of population 2.
    #---------------------------
    x = x_pop[2,:]  
    m3, v3, s3, k3 = calculates_statistical_quantities
    _of_a_population(x, x2, Bins - 1, population_size[2])  
    m3 = round(m3, 3) 
    v3 = round(v3, 3) 
    s3 = round(s3, 3)     
    k3 = round(k3, 3) 
\end{python}
\end{roundedcodebox}
  \column{0.48\textwidth}  % First column
    \begin{roundedcodebox}
    \begin{python}  

\end{python}
\end{roundedcodebox}
    \end{columns}
\end{frame}

%----------------------------------------------------------------------------------------
\begin{frame}[fragile]
    \frametitle{Python}
\begin{columns}
  \column{0.48\textwidth}  % First column
    \begin{roundedcodebox}
    \begin{python}  
    
\end{python}
\end{roundedcodebox}
  \column{0.48\textwidth}  % First column
    \begin{roundedcodebox}
    \begin{python}  

\end{python}
\end{roundedcodebox}
    \end{columns}
\end{frame}

%----------------------------------------------------------------------------------------

% \begin{frame}[fragile]
%     \frametitle{C}
    
%     \begin{clang}
% #include <stdio.h>

% int main() {
%     int numero = 5;
%     int dobro = 2 * numero;
    
%     printf("O dobro de %d eh %d\n", numero, dobro);
%     return 0;
% }
%     \end{clang}
% \end{frame}

% %----------------------------------------------------------------------------------------
% \begin{frame}[fragile]
%     \frametitle{C++}
    
%     \begin{cpp}
% #include <iostream>
% using namespace std;

% int main() {
%     int numero = 5;
%     int dobro = 2 * numero;
    
%     cout << "O dobro de " << numero;
%     cout << " eh " << dobro << endl;
%     return 0;
% }
%     \end{cpp}
% \end{frame}

% %----------------------------------------------------------------------------------------
% \begin{frame}[fragile]
%     \frametitle{R}
    
%     \begin{rlang}
% # Função para calcular o dobro
% calcular_dobro <- function(x) {
%   return(2 * x)
% }

% # Testando a função
% numero <- 5
% resultado <- calcular_dobro(numero)
% print(paste("O dobro de", numero, "é", resultado))
%     \end{rlang}
% \end{frame}

% %----------------------------------------------------------------------------------------

% \begin{frame}[fragile]
%     \frametitle{Java}
    
%     \begin{java}
% public class Exemplo {
%     public static void main(String[] args) {
%         int numero = 5;
%         int dobro = 2 * numero;
        
%         System.out.println("O dobro de " + numero +
%                          " eh " + dobro);
%     }
% }
%     \end{java}
% \end{frame}